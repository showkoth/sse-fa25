\documentclass[12pt]{article}
\usepackage[utf8]{inputenc}
\usepackage[margin=1in]{geometry}
\usepackage{listings}
\usepackage{xcolor}
\usepackage{fancyhdr}
\usepackage{titlesec}
\usepackage{enumitem}

% Code listing settings
\lstset{
    basicstyle=\ttfamily\small,
    breaklines=true,
    frame=single,
    backgroundcolor=\color{gray!10},
    keywordstyle=\color{blue},
    commentstyle=\color{green!60!black},
    stringstyle=\color{red},
    showstringspaces=false,
    tabsize=4,
    captionpos=b
}

% Python code style
\lstdefinestyle{python}{
    language=Python,
    basicstyle=\ttfamily\small,
    keywordstyle=\color{blue},
    commentstyle=\color{green!60!black},
    stringstyle=\color{red},
    backgroundcolor=\color{gray!10},
    frame=single,
    breaklines=true,
    showstringspaces=false
}

% HTML code style
\lstdefinestyle{html}{
    language=HTML,
    basicstyle=\ttfamily\small,
    keywordstyle=\color{blue},
    commentstyle=\color{green!60!black},
    stringstyle=\color{red},
    backgroundcolor=\color{gray!10},
    frame=single,
    breaklines=true,
    showstringspaces=false
}

% Header and footer
\pagestyle{fancy}
\fancyhf{}
\rhead{Part 2: Finding Issues}
\lhead{HW2 - Security Vulnerability Analysis}
\cfoot{\thepage}

% Title formatting
\titleformat{\section}{\Large\bfseries}{\thesection}{1em}{}
\titleformat{\subsection}{\large\bfseries}{\thesubsection}{1em}{}

\begin{document}

\title{\textbf{Home Work 2 \\ \large Part 2: Finding Issues}}
\author{Showkot Hossain}
\date{}
\maketitle

\section{Q1: Vulnerability Analysis}

I identified \textbf{4 critical security vulnerabilities} in the Django task management web application:

\vspace{0.5cm}
\hrule
\vspace{0.5cm}

\subsection{Vulnerability 1: SQL Injection}

\textbf{Location:} \texttt{tasktracker/views.py}, line 37 in \texttt{add()} function

\textbf{Vulnerable Code:}
\begin{lstlisting}[style=python]
cursor.executescript(f"INSERT INTO tasktracker_task(user_id, status, due_date, title) VALUES ({request.user.id},'{status}', '{due_date}', '{title}')")
\end{lstlisting}

\textbf{Exploitation:} User input is directly concatenated into SQL queries without sanitization. An attacker can inject malicious SQL through form fields.

\textbf{Example:} Setting title to \texttt{'; DROP TABLE tasktracker\_task; --} would delete the entire task table.

\textbf{CWE-ID:} CWE-89 (SQL Injection)

\vspace{0.5cm}
\hrule
\vspace{0.5cm}

\subsection{Vulnerability 2: Cross-Site Scripting (XSS)}

\textbf{Location:} \texttt{tasktracker/templates/index.html}, line 11

\textbf{Vulnerable Code:}
\begin{lstlisting}[style=html]
<b>Task #{{t.id}}: {{t.title | safe}}</b>
\end{lstlisting}

\textbf{Exploitation:} The \texttt{safe} filter disables HTML escaping, enabling stored XSS attacks. An attacker can create a task with malicious JavaScript in the title.

\textbf{Example:} Task title \texttt{<script>alert('XSS')</script>} executes when users view the task list, potentially stealing cookies or hijacking sessions.

\textbf{CWE-ID:} CWE-79 (Cross-Site Scripting)

\vspace{0.5cm}
\hrule
\vspace{0.5cm}

\subsection{Vulnerability 3: Insecure Direct Object Reference (IDOR)}

\textbf{Location:} \texttt{tasktracker/views.py}, lines 51-56 in \texttt{delete()} function

\textbf{Vulnerable Code:}
\begin{lstlisting}[style=python]
def delete(request, pk):
    task = Task.objects.get(id = pk)
    task.delete()
    return HttpResponseRedirect(reverse(f'tasktracker:index'))
\end{lstlisting}

\textbf{Exploitation:} The function doesn't verify task ownership. Any authenticated user can delete other users' tasks by manipulating the URL parameter.

\textbf{Example:} User can access \texttt{/tasktracker/delete/1/} to delete any task with ID 1, regardless of ownership.

\textbf{CWE-ID:} CWE-639 (Authorization Bypass Through User-Controlled Key) \\
(Also related: \textbf{CWE-862:} Missing Authorization)

\vspace{0.5cm}
\hrule
\vspace{0.5cm}

\subsection{Vulnerability 4: Hard-coded Secret Key}

\textbf{Location:} \texttt{website/settings.py}, line 23

\textbf{Vulnerable Code:}
\begin{lstlisting}[style=python]
SECRET_KEY = '#n6kks0cs$a-7k*67)k(nof$7z6&l+u97ea)nl_r8frg_mrmd1'
\end{lstlisting}

\textbf{Exploitation:} The Django secret key is exposed in source code. Attackers can use this key to forge session cookies, bypass CSRF protection, and generate valid password reset tokens.

\textbf{Example:} With the secret key, an attacker can impersonate any user by crafting valid Django session data.


\textbf{CWE-IDs:} \\ 
- \textbf{CWE-321: Use of Hard-coded Cryptographic Key} (most precise) \\ 
- \textbf{CWE-798: Use of Hard-coded Credentials} (also related)

\vspace{0.5cm}
\hrule
\vspace{0.5cm}

\end{document}